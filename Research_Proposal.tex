\chapter{Research Proposal}

This appendix details significant changes that exist between this thesis and the initial research proposal submitted in Semester 1.

There were some significant changes made to the project after the Research Proposal submission. The broad area of interest remained the same: namely, optimising strain estimation methods for use in the assessment of surgical margins in breast conserving surgery. However, a paradigm shift occurred from attempting to improve upon the image quality, to attempting to improve the speed with which the image could be produced. This thesis is hence structured to look at speeding up the strain estimation process first, to enable real-time imaging, and only secondly to ensure the image quality was maintained.
 
The analysis on image quality focused on removing noise by simply applying a broad lateral smoothing filter, which did not induce any significant time expense on the processing. In comparison to this, initial investigation was made into attempting to differentiation between optical noise, using a statistical analysis of the OCT signal as a random phasor sum, and mechanical noise that represented physical features in the sample. The idea was, that these two types of 'artefacts' in the image, would have very different noise signatures, where the mechanical noise artefact could be seen as not noise at all, but the image itself. 
The reason further investigation was not carried into this area, was because of the real-world implications for the imaging technique - the end goal of the research group, which obviously had a large impact on the project direction, was to utilise the OCE technology to provide intra-operative surgical margin assessment. The problem with techniques based on statistical analysis of noise, among others, is that the trade off between the high processing time cost and the very small addition to image quality that they would have (as discovered by some initial investigations soon after the proposal was submitted) was not necessarily a good enough justification, keeping in mind the overall aim of the project. 

\includepdf[pages=-]{Emily_Hackett_Research_Proposal.pdf}
