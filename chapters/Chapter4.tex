\chapter{Results}


\section{Data Acquisition}
A silicone phantom was imaged using (...)

The strain estimation algorithms described following utilise this calculated averaged phase difference as the beginning step of the processing pipeline, and assume that the phase difference data is already loaded into memory.

\section{Strain Estimation Algorithms}
A total of six different strain estimation algorithms were implemented and tested, and can be differentiated based on how the phase unwrapping issue is addressed, and the digital differentiation filter used. 

\subsection{WLS with Unwrapped Phase}
The WLS with phase unwrapping algorithm unwraps the phase by reading in the entire volume and saves the unwrapped phase data to file. In a sequential loop over each B-scan, wrapped in a Matlab parfor parallel, the process reads in the complex phase and unwrapped phase for each B-scan. If lateral smoothing has been specified, it smooths the unwrapped phase using unweighted Gaussian smoothing by convoluting the unwrapped phase with the 1D lateral filter. A sequential looping is then performed over each A-scan and all fit segments within the given B-scan, which calculates the WLS estimate of strain using summations and the weights from the complex phase data.

\subsection{SG Filtering on Unwrapped Phase}
This function unwraps the phase by the same process specified above, which requires a read from file of the unwrapped phase for each B-scan. If lateral smoothing required, the unwrapped phase is similarly smoothed via convolution with an unweighted Gaussian smoothing filter. To estimate the strain, this smoothed unwrapped phase data is convolved with the analytically derived Savitzky-Golay filter in a single action of the entire B-scan.

\subsection{SG Filtering on Offset Phase}
In order to correct for phase unwrapping, the function reads in the complex phase for each B-scan (in a parallel Matlab parfor loop) and loops over all fit segments within the B-scan, taking a matrix of values across all A-scans for the given depths. Using Matlab's inbuilt bsxfun operator, an averaged complex phase value for each fit segment is calculated and divided (corresponding to subtraction of the phase) to move the phase values into an unwrapped region. The Savitzky-Golay filter, which was pre-calculated using the analytical solution, is then convolved with the matrix in the axial direction to produce the strain estimation values at that given depth. 

\subsection{FD on Weighted Smoothing of Phase Difference}


\subsection{FD on Unweighted Smoothing of Phase Difference}

\subsection{Pre-filtered Phase Difference with Smoothed FD Strain}


\section{Phantom Results for Strain Elastograms}

On the basis of these results, it was decided to see if implementing lateral averaging over the separated B-scans could improve the sensitivity of the lower-order differentiation techniques towards that of the WLS. 

\section{Phantom Results for Laterally Averaged Strain Elastograms}

\section{Analysis of Image Resolution}
It was found that the process of fitting a step response to the object boundaries was impossible for strain elastograms with no lateral averaging, due to the noisiness of the data. The comparison between a lateral segment with no axial smoothing for the unweighted finite difference (with a fit resoliution of ??), and the same segment smoothed with a lateral resolution of (??) and their corresponding fitted error functions is shown in Figure ??.
