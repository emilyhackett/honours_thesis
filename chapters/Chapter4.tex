\chapter{Results}

\section{Assessment of Strain Processing}
\subsection{Processing Speed}
The processing speed of the different strain estimation algorithms was assessed using both the processing of a single B-scan, and for a complete 3D volume. In addition, differences in algorithm complexity are pointed out between the methods, as well as their ease with which they can be parallelised, and the likelihood of a speed up made possible by parallelising the processing.
\subsection{Sensitivity}
\subsection{Image Resolution}


\section{Strain Estimation Algorithms}
A total of six different strain estimation algorithms were implemented and tested, and can be differentiated based on how the phase unwrapping issue is addressed, and the digital differentiation filter used. 
\subsection{WLS with Unwrapped Phase}
\subsection{SG Filtering on Unwrapped Phase}
\subsection{SG Filtering on Offset Phase}
\subsection{FD on Weighted Smoothing of Phase Difference}
\subsection{FD on Unweighted Smoothing of Phase Difference}
\subsection{Pre-filtered Phase Difference with Smoothed FD Strain}

\section{Data Acquisition}
A silicone phantom was imaged using (...)

\section{Phantom Results for Strain Elastograms}

On the basis of these results, it was decided to see if implementing lateral averaging over the separated B-scans could improve the sensitivity of the lower-order differentiation techniques towards that of the WLS. 

\section{Phantom Results for Laterally Averaged Strain Elastograms}

