\chapter{The Physics of OCE}

Optical coherence elastography (OCE) can be thought of as the combination of a foundational imaging technique, optical coherence tomography, and its application to imaging mechanical properties: elastography. These two pillars of OCE are examined first separately, in Sections 2.1 and 2.2, then as connected components of a composite OCE system in Section 2.3. In this section, the characteristics of standard OCE systems are discussed, with an eye to understanding the implications of the phase-sensitive, compression OCE system used in this project.

\section{Optical Coherence Tomography}
Optical coherence tomography (OCT) can be thought of as the optical equivalent to ultrasound, in that it detects the light reflected from scatterers within a tissue. The reflected light interferes with a reference beam, enabling depth-resolved reconstruction of the location of the scatterers within the sample \cite{chin_parametric_2016}. Because the speed of light in tissue is significantly higher than that of sound, OCT offers a much higher resolution than ultrasound, on the order of $5-15 \mu m$ \cite{kennedy_emergence_2017}, as limited by the coherence length of the light source \cite{huang_optical_1991}. However, this higher resolution comes at the expense of depth of penetration into the tissue, which for OCT, is only $1-2mm$ beneath the surface \cite{schmitt_optical_1999}, in comparison to the relatively deep imaging capabilities of ultrasound.

Time-domain OCT uses a simple low coherence Michelson interferometer set up, where the interference signal is measured between the light reflecting from the tissue, and an scanned reference mirror that changes the depth imaged \cite{huang_optical_1991}. The scanning of the reference mirror to image into the tissue produces a 1D A-scan, that contains information about the detected irradiance of the interferometer as a function of depth. Taking multiple A-scans across by moving laterally across a sample surface produces a 2D B-scan. Multiple B-scans can be used as cross-sections to build up an entire 3D scan volume, otherwise known as a C-scan, a process which can be seen in Figure 2. These descriptions are taken from those conventionally utilised in ultrasound imaging. The disadvantage of this set up however is the expensive acquisition times for imaging of 3D volumes.

Fourier-domain OCT systems remove the need for a scanning mirror, and allows the reproduction of a single A-scan with one detection event based on principles of spectral interferometry \cite{chin_parametric_2016}. Rather than detecting the reflected intensity as a function of depth as in time-domain OCT, it is detected as a function of wavenumber of a broadband or swept frequency light source. The Wiener-Khinchin theorem dictates that the inverse Fourier transform of this measured spectral density provides the complex coherence function at different depths along the A-scan line \cite{schmitt_optical_1999}. The speed up using Fourier-domain OCT over time-domain OCT allows the imaging of 3D volumes, of approximately $10mm \times 10mm \times 2mm$ to be acquired in less than 1 second \cite{kennedy_emergence_2017}. The benefit of Fourier-domain OCT over time-domain is not only in allowing a much faster acquisition time, but it also produces a complex signal that carries phase information, allowing it to be used as to quantify displacement within a sample, as described in Section 1.3.

\section{Optical Coherence Elastography}
Many elastography methods based on optical imaging techniques exist, and can be differentiated between by examining how the mechanical load is applied to the tissue, as well as what parameter is utilised to form the image, or 'elastogram'. These different factors, as well as the underlying imaging technique, determine the resolution and penetration of the resulting imaging. In addition these parameters may be quantitative or qualitative, which has implications for diagnostic capability, and mechanical loading methods can be simple or complex, which introduces limitations on image acquisition time.

Elastography techniques commonly used with optical imaging modalities, among others, can be thought of in three categories: compressive, resonant and shear wave based elastography. Static or quasi-static methods of loading include compressing the tissue, either through an indentation point or in bulk, and measures the resulting strain in the tissue by detecting the displacement vector produced between two image acquisitions \cite{kennedy_optical_2014}. This displacement can be used to calculate the local strain at different points in the image, to which elasticity is inversely related. Resonant loading methods, that impart a step load using acoustic radiation force or magnetomotive forces record the natural frequency of the tissue, whose square is directly proportional to elasticity \cite{kennedy_optical_2015}. Shear wave techniques use the phase velocity of a propagating wave within the tissue as a mechanical contrast parameter, as generated by a pulsed or periodic load. 

On top of these highly variable loading methods, the detection of tissue deformation can be done using speckle tracking via cross-correlation algorithms (which are highly influenced by speckle decorrelation noise), or by phase-sensitive detection \cite{kennedy_strain_2012}.

Optical coherence elastography (OCE) is the combination of elastography with an OCT imaging system, as first proposed by Schmitt in 1998 \cite{schmitt_oct_1998}. The micro-architecture of tissue carries important information about disease states, however these are poorly demonstrated using only optical contrast. Using elastography with optical coherence tomography as the underlying imaging modality allows high mechanical contrast imaging on scales previously inaccessible to ultrasound and MRI-based elastography techniques. Of particular interest in this project is compressive OCE utilising phase-sensitive detection. 

Compression OCE produces a displacement field in the tissue, that can be measured in order to calculate the strain present. The local strain is estimated as the spatial derivative of these displacements with respect to depth \cite{kennedy_review_2014}. For phase-sensitive compression OCE, this displacement is directly related to the phase difference between an unloaded (not compressed) and loaded (compressed) scan, by Equation 2 \cite{kennedy_strain_2012}. However compressive loading techniques provide only qualitative comparison in strain elastograms, since it is dependent on the amount of compressive loading applied to the tissue. It is possible to produce quantitative elastogram images by measuring the stress locally applied to the sample by introducing a known stress layer above the imaged sample, as discussed in detail in \cite{kennedy_quantitative_2015}, however this will not be discussed further here. The benefit of compression techniques is their fast acquisition speeds for entire 3D volumes, as well as their ability to be extended to needle based OCE systems for imaging deep within the tissue \cite{kennedy_review_2014}. 

\begin{equation}
u_i = \frac{\phi_i^{loaded}-\phi_i^{unloaded}}{4\pi n} = \frac{\Delta\phi_i}{4\pi n}
\end{equation}

OCE inherits the high spatial resolution of the underlying OCT imaging system, however the strain estimation process that is applied to the phase-sensitive data has the potential to degrade this slightly, but still allowing imaging of tissue microarchitecture. However OCE also maintains the low penetration depth of OCT imaging, making the need for intra-operative OCE set ups, capable of scanning deep tissue temporarily exposed in surgery, even more essential. 

\section{Phase-Sensitive Displacement Measurement}

The efficiency of any elastography imaging technique is dependent on its ability to measure the deformation of the tissue in response to the mechanical load accurately. Phase-sensitive displacement measurement in OCE is enabled by Fourier-domain OCT systems, and allows tissue displacements on the order of nanometres to be detected by the system, corresponding to microstrain sensitivity \cite{kennedy_review_2014}. However phase-sensitive OCE methods bring with them their own problems, one being that te detected phase can only take values in the interval $[0,2\pi]$. This results in wrapping of the phase difference and if uncorrected, will result in discontinuities in the calculated displacement field.


%% PHASE UNWRAPPING NOT INTEGRATED:
% Should discuss Zaitsev method in detail here, as well as reference & describe
% 3D phase unwrapping algorithm implemented by Lixin
The issue of phase unwrapping is tackled using three different approaches. Firstly, the current phase unwrapping algorithm using averaging of en face images over a 3D volume is used unchanged, and then the resulting strain estimation using the reconstructed displacements is performed using a convolution filter to maximise speed. 

Secondly, the phase unwrapping algorithm is dropped, and a phase offset is subtracted from each fit segment in the complex domain to remove a possible wrapping event (by returning it to an unwrapped region). From here the strain is estimated using the unwrapped fit segment. However unlike the volume phase unwrapping algorithm, this does not re-produce the displacement field and therefore a differentiation filter cannot be applied to the entire scan at once. Subtraction of the phase offset is a non-linear filter operation, and therefore must be performed on each fit segment separately. This bottleneck slows down the strain estimation process for methods that use a phase offset to unwrap, however it can be mitigated by methods described in the next section.

The third approach does not tackle the issue of phase unwrapping directly, but rather makes the fit segment so small it is assumed no wrapping can occur between. Using finite difference to calculate the strain, performed on the complex phase, ensures that no wrapping artefacts enter the image, provided that no wrapping events occur between consecutive pixels, which is a valid assumption with the amount of compression used. The down side to this technique is that finite difference is a noisy differentiation method, resulting in significant degradation of image quality when used alone. Ways to compensate for loss of sensitivity are discussed in the section below.

