\chapter{Background}

\section{Optical coherence tomography}
OCE uses optical coherence tomography (OCT) as the underlying technique for imaging tissue. OCT is the optical analog of ultrasound, in that it uses backscattered, or reflected, light from scatterers within the tissue to form a depth resolved image. 

\section{Elastography}
Elastography produces images of the mechanical properties of tissue, characterised by three deciding parameters: the underlying imaging technique, the method of loading, and the resulting measured parameter. 

\section{Optical coherence elastography}
OCE is the combination of elastography with an OCT imaging system. 


\section{Strain Estimation}
Imaging strain is one application of OCE, which produces qualitative images of the stiffness of tissue. The tissue strain is defined as the amount of displacement induced in the tissue by the mechanical loading per spatial region. It is the spatial derivative of the displacement imaged by the OCE system.

\subsection{Subsection}
