\chapter{Discussion}

\begin{itemize}
\item \ac{uwwls}
\item \ac{uwsg}
\item \ac{posg}
\item \ac{wfd}
\item \ac{uwfd}
\item \ac{pffd}
\end{itemize}

The optimal strain estimation technique in this context appears to be the \ac{uwfd} algorithm, applied with lateral averaging. At a fit resolution of $40\mu m$ and a lateral smoothing resolution of $40\mu m$, all image quality parameters are improved in comparison with the previously standard \ac{uwwls} approach (\autoref{comparison_table}).

\begin{table}[h]
	\begin{tabularx}{\textwidth}{Yrrc}
		\toprule
		& \textbf{\ac{uwwls}} & \textbf{\ac{uwfd}} & \textbf{Improved?}\\
		\midrule 
		Strain Filter FWHM ($\mu$m) & 70 & 40 \\
		Lateral Smoothing Filter FWHM ($\mu$m) & 0 & 20 \\
		\midrule
		2D Processing Time (s) & & \\
		3D Processing Time (s) & 153.67 & \\
		Strain Sensitivity (m$\epsilon$) & 0.1607 & 0.1385 & 16\% \\
		Axial Image Resolution FWHM ($\mu$m) & 315.87 & 258.37 & \textbf{20\%} \\
		Lateral Image Resolution FWHM ($\mu$m) & 131.33 & 115.98 & 13\% \\
		\bottomrule
	\end{tabularx}
	\caption{Comparison of the previously standard \ac{uwwls} strain estimation technique and the optimised \ac{uwfd}.}
	\label{comparison_table}
\end{table}

\section{Future Work}

It has been discovered that \ac{gpu}s can greatly accelerate the processing of strain imaging for techniques such as ultrasound elastography that utilise speckle-tracking (rather than phase-sensitive measurement) \cite{peng_gpu-accelerated_2017}, therefore there is reason to believe this could also benefit in the phase-sensitive \ac{oce} case. Of particular benefit is the fact that all four strain estimation techniques that do not involve the volume phase unwrapping algorithm, are capable of processing individual B-scans completely independently, making it very likely that introducing parallel processing to these methods would significantly decrease the computation time. Applying filter convolutions (as in \ac{sg} filtering, and Gaussian smoothing in the \ac{fd} case) in parallel and on \ac{gpu}s could further improve the processing speed, towards intra-operative time frames, and even video-rate imaging.

Before these possibilities are investigated however, the algorithms will likely need to be moved out of the Matlab interface, and applied in another language more suited to fast numerical processing.

Further work also must be done in translating this strain estimation algorithm into the clinic, by analysing its performance on a range of breast tissue samples. In particular, before being able to apply this technique to assess cancer margins during breast conserving surgery, diagnostic specificity and sensitivity (the ability to accurately differentiate healthy tissue from tumour) of the technique must be analysed using large case studies.

In conjunction with this, research also could be done in applying these strain estimation techniques to quantitative elastography imaging, to determine which one is optimal. Measurement of a stress using a stress layer is relatively fast, and most of the computational burden falls on the strain estimation, therefore strain estimators that optimise the processing speed would benefit in this area also. 

