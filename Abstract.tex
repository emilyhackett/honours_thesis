\begin{abstract}

	Re-excision rates for breast conserving surgery, as a result of inaccurate tumour detection during surgery, are estimated to be 30\% in 2017. There is a significant need for fast imaging techniques capable of assessing tumour margins in breast tissue with high resolution and within intra-operative time frames. Cancer often causes changes in the mechanical properties of tissue, which elastography can image. Elastography describes a family of imaging techniques capable of imaging these changes in mechanical properties by examining their response to mechanical loading. \ac{oce} refers to a number of elastography techniques that utilise \ac{oct} as the underlying imaging technique, and are therefore capable of producing images of mechanical properties (such as stiffness, or strain response), on the micrometre resolution scale, with mm to cm field of view size, however only at shallow (mm) penetration depths. To make use of \ac{oce} in intra-operative surgical margin assessment for breast cancer, there is a need to produce images of strain rapidly. In order to do this, fast strain estimation algorithms must be developed, that maintain high image quality in terms of sensitivity and image resolution. 
	
	Six strain estimation algorithms, differentiated by their phase unwrapping technique and the strain estimate applied, are implemented and analysed on a silicon phantom, using the metrics of processing time, strain sensitivity and image resolution. It was found that strain estimation techniques utilising Gaussian smoothing with finite difference were the fastest, at 50$\times$ faster than previous techniques, and had maintained similar sensitivity. Applying lateral averaging on top of the strain estimation further improved the sensitivity, however at the cost of axial and lateral image resolution. 

	Using an unweighted Gaussian filter on the phase difference, followed by finite difference, produced the optimum combination of processing speed, sensitivity and image resolution, with improvements made upon all. Further work is suggested in terms of implementing the algorithms in parallel, possibly on \ac{gpu}s, and further extension into quantitative elastography.
	
\end{abstract}


