\chapter{Results}


\section{Data Acquisition}
A silicone phantom was imaged using (...)

The strain estimation algorithms described following utilise this calculated averaged phase difference as the beginning step of the processing pipeline, and assume that the phase difference data is already loaded into memory.

\section{Strain Estimation Algorithms}

\begin{figure}[h]
\begin{tikzpicture}[node distance=2cm, remember picture]
	\centering
	\node [anchor=north] at (0,0) (input) [io] {INPUT: $\Delta\phi_i$};
    
    \node (volume_unwrap) at (-4,-3) [basic] {Unwrapping Algorithm};
    \draw [wls] (input.225) edge[out=270,in=90] (volume_unwrap.135);
    \draw [uwsg] (input.240) edge[out=270,in=90] (volume_unwrap.120);
    
    \node (bscan_loop) at (0,-5.5) [main_long] {LOOP OVER B-SCANS};
    \draw [wls] (volume_unwrap.225) edge[out=270,in=90] (bscan_loop.135);
    \draw [uwsg] (volume_unwrap.240) edge[out=270,in=90] (bscan_loop.120);
    \draw [posg] (input.260)--(bscan_loop.100);
    \draw [wfd] (input.280)--(bscan_loop.80);
    \draw [uwfd] (input.300)--(bscan_loop.60);
    \draw [fdsm] (input.315)--(bscan_loop.45);
    
   	\node (variance) at (-4,-8) [basic] {Calculate variances};
    \draw [wls] (bscan_loop.225) edge[out=270,in=90] (variance.135);
    
    \node (unweighted) at (4,-8) [basic] {Remove phase weighting};
    \draw [uwfd] (bscan_loop.300) edge[out=270,in=90] (unweighted.60); 
    
    \node (lateral) at (0,-10.5) [temp] {LATERAL SMOOTHING};
    \draw [uwsg] (bscan_loop.240)--(lateral.120);
    \draw [posg] (bscan_loop.260)--(lateral.100);
    \draw [wfd] (bscan_loop.280)--(lateral.80);
    \draw [fdsm] (bscan_loop.315)--(lateral.45);

	\draw [wls] (variance.225) edge[out=270,in=90] (lateral.135);
    \draw [uwfd] (unweighted.300) edge[out=270,in=90] (lateral.60);
    
    \node (fit_loop) at (-4,-13.5) [main] {LOOP OVER FIT SEGMENTS};
    \draw [wls] (lateral.225) edge[out=270,in=90] (fit_loop.135);
    \draw [posg] (lateral.260) edge[out=270,in=90] (fit_loop.100);
    
    \node (ascan_loop) at (-6.5,-16) [main] {LOOP OVER A-SCANS};
    \draw [wls] (fit_loop.225) edge[out=270,in=90] (ascan_loop.135);
    
    \node (wls_strain) at (-4,-18) [basic] {WLS strain estimation};
    \draw [wls] (ascan_loop.225) edge[out=270,in=90] (wls_strain.135);
    
    \node (phase_offset) at (-2.5,-16) [basic] {Divide complex phase offset};
    \draw [posg] (fit_loop.260) edge[out=270,in=90] (phase_offset.100);
    
    \node (sg_filter) at (0,-18) [basic] {Convolve with SG Filter};
    \draw [posg] (phase_offset.260) edge[out=270,in=90] (sg_filter.100);
    \draw [uwsg] (lateral.240) edge[out=270,in=90] (sg_filter.120);
   
   \node (gauss_filter) at (2,-15.8) [basic] {Convolve with an axial 1D Gaussian filter};
   \draw [wfd] (lateral.280) edge[out=270,in=90] (gauss_filter.80);
   \draw [uwfd] (lateral.300) edge[out=270,in=90] (gauss_filter.60);
   
   \node (pre_filter) at (6,-15.5) [basic] {Convolve with a small 2D Gaussian 'pre-filter'};
   \draw [fdsm] (lateral.315) edge[out=270,in=90] (pre_filter.45);
   
   \node (fd) at (4,-18) [basic] {FD strain estimation};
   \draw [wfd] (gauss_filter.280) edge[out=270,in=90] (fd.80);
   \draw [uwfd] (gauss_filter.300) edge[out=270,in=90] (fd.60);
   \draw [fdsm] (pre_filter.315) edge[out=270,in=90] (fd.45);
   
   \node (output) at (0,-20.5) [io] {OUTPUT: $\epsilon_z$};
   \draw [wls] (wls_strain.225) edge[out=270,in=90] (output.135);
   \draw [uwsg] (sg_filter.240) edge[out=270,in=90] (output.120);
   \draw [posg] (sg_filter.260) edge[out=270,in=90] (output.100);
   \draw [wfd] (fd.280) edge[out=270,in=90] (output.80);
   \draw [uwfd] (fd.300) edge[out=270,in=90] (output.60);
   \draw [fdsm] (fd.315) edge[out=270,in=90] (output.45);

\end{tikzpicture}
\label{flowchart}
\end{figure}

A total of six different strain estimation algorithms were implemented and tested, and can be differentiated based on how the phase unwrapping issue is addressed, and the digital differentiation filter used. Figure \ref{flowchart} demonstrates the decision processes in these strain estimation algorithms that separate the way they get from the complex phase difference delivered by the OCT system and pre-processing, to the final strain elastogram image.
The functions worked by using the complex phase difference C-scan as an input, as well as the specifications for the fit (axial) and lateral resolutions of the system. The fit resolution of the system determined the fit window for the WLS and Savitzky-Golay based strain estimation techniques, and the axial Gaussian $\sigma$ value for the finite difference based techniques. The lateral resolution determined to what extent the strain images were laterally averaged by a Gaussian smoothing filter (for reasons discussed in the later section 4.3). If a lateral resolution of 0 was specified, no lateral smoothing was performed. 

\subsection{WLS with Unwrapped Phase}
The WLS with phase unwrapping algorithm unwraps the phase by reading in the entire volume and saves the unwrapped phase data to file. In a sequential loop over each B-scan, wrapped in a Matlab parfor parallel, the process reads in the complex phase and unwrapped phase for each B-scan. If lateral smoothing has been specified, it smooths the unwrapped phase using unweighted Gaussian smoothing by convoluting the unwrapped phase with the 1D lateral filter. A sequential looping is then performed over each A-scan and all fit segments within the given B-scan, which calculates the WLS estimate of strain using summations and the weights from the complex phase data.

\subsection{SG Filtering on Unwrapped Phase}
This function unwraps the phase by the same process specified above, which requires a read from file of the unwrapped phase for each B-scan. If lateral smoothing required, the unwrapped phase is similarly smoothed via convolution with an unweighted Gaussian smoothing filter. To estimate the strain, this smoothed unwrapped phase data is convolved with the analytically derived Savitzky-Golay filter in a single action of the entire B-scan.

\subsection{SG Filtering on Offset Phase}
In order to correct for phase unwrapping, the function reads in the complex phase for each B-scan (in a parallel Matlab parfor loop) and loops over all fit segments within the B-scan, taking a matrix of values across all A-scans for the given depths. Using Matlab's inbuilt bsxfun operator, an averaged complex phase value for each fit segment is calculated and divided (corresponding to subtraction of the phase) to move the phase values into an unwrapped region. The Savitzky-Golay filter, which was pre-calculated using the analytical solution, is then convolved with the matrix in the axial direction to produce the strain estimation values at that given depth. 

\subsection{FD on Weighted Smoothing of Phase Difference}
The complex weighted phase difference that acts as the function's input is smoothed using an axial Gaussian filter dependent on the fit resolution supplied (and a lateral filter if specified). For strain estimation, finite difference is performed using bsxfun on the complex phase data (in order to prevent the introduction of  wrapping artefacts), by dividing one pixel by the conjugate of the previous pixel in order to subtract the phase values. This estimate of the spatial derivative of the phase difference is then converted into strain values.

\subsection{FD on Unweighted Smoothing of Phase Difference}
This algorithm works almost identically to the one described in the previous section, except that prior to using a Gaussian smoothing filter on the complex phase data, the amplitude is normalised to 1 for all phase difference values, effectively removing the optical weighting that comes from the OCT signal. From here the finite difference is calculated, and from this the strain. 

\subsection{Pre-Filtered Phase Difference with Smoothed FD Strain}
The 'Pre-Filtered' phase difference algorithm was implemented as an effort to remove 'ripples' introduced by high intensity pixels in the finite difference with weighted smoothing described in Section 4.2.4, which will be discussed later in the chapter. Using the weighted complex phase difference data, a small 2D Gaussian filter is applied to smooth out high intensity 'speckle' pixels. This pre-filter is set at $20 \mu m$ resolution, approximately on the order of the speckle size. This action aims to counter the disproportionate influence of high intensity pixels. From here, the strain is estimated using a finite difference, and the resulting strain is smoothed using a 2D Gaussian filter to match the required fit and lateral resolution values. Since the pre-filter contributes to the fit and lateral resolutions, for any given fit resolution $\text{FR}_{total}$ and lateral resolution $\text{LR}_{total}$ supplied to the function as input parameters, the resulting Gaussian smoothing resolutions must take into account the contribution from the pre-filter as in Equations \ref{prefilter_res_FR} and \ref{prefilter_res_LR} below.

\begin{equation}
	\text{FR}_{Gauss} = \sqrt{\text{FR}_{total}^2 - \text{FR}_{pre}^2}
    \label{prefilter_res_FR}
\end{equation}

\begin{equation}
    \text{LR}_{Gauss} = \sqrt{\text{LR}_{total}^2 - \text{LR}_{pre}^2}
    \label{prefilter_res_LR}
\end{equation}

In order to maintain consistency between the strain estimation techniques, for the case in which no lateral smoothing is applied for any other algorithms, the pre-filter becomes 1D Gaussian smoothing filter only, so that $\text{LR}_{pre}\rightarrow 0$. 

\section{Phantom Strain Elastogram Results}

Figure \ref{bscan_images_1} shows the resulting B-scan strain images, taken from the centre of the phantom, for the six different strain estimation techniques described above. 



On the basis of these results, it was decided to see if implementing lateral averaging over the separated B-scans could improve the sensitivity of the lower-order differentiation techniques towards that of the WLS. 

\section{Phantom Strain Elastogram Results with Lateral Averaging}

\section{Analysis of Image Resolution}
It was found that the process of fitting a step response to the object boundaries was impossible for strain elastograms with no lateral averaging, due to the noisiness of the data. The comparison between a lateral segment with no axial smoothing for the unweighted finite difference (with a fit resoliution of ??), and the same segment smoothed with a lateral resolution of (??) and their corresponding fitted error functions is shown in Figure ??.


